\documentclass[a4paper, 14pt]{extarticle}
  \usepackage{cmap}
  \usepackage[hidelinks,pdftex,unicode]{hyperref}
  \usepackage{mathtext} % для кириллицы в формулах
  \usepackage[T2A]{fontenc}
  \usepackage[utf8]{inputenc}
  \usepackage[english,russian]{babel}
  \usepackage{indentfirst}
  \usepackage{cite}
  \usepackage{amsmath} % для \eqref
  \usepackage{amssymb} % для \leqslant
  \usepackage{amsthm} % для \pushQED
  \usepackage[usenames,dvipsnames]{xcolor}
  \usepackage[pdftex]{graphicx}
  \usepackage{subfig}
  \usepackage{numprint}
  \usepackage[left=20mm,right=20mm,top=20mm,bottom=25mm,bindingoffset=0cm]{geometry}
  \usepackage{datetime}
  \usepackage{setspace} % для иных отступов в tikz-картинках
  \usepackage{tikz}
    \usetikzlibrary{positioning,fit,shapes,calc}
  \usepackage[babel,protrusion=true,expansion]{microtype}
  \usepackage{nameref}
  \graphicspath{{../img/}{../../img/}}
  \frenchspacing

  \DeclareSymbolFont{T2Aletters}{T2A}{cmr}{m}{it} % кириллица в формулах курсивом

  \addto\captionsrussian{
    \renewcommand\contentsname{Содержание}
    % перекрываю \refname, чтобы список литературы сам добавлял себя в оглавление
    \let\oldrefname\refname
    \renewcommand\refname{\addcontentsline{toc}{section}{\oldrefname}\oldrefname}
  }

  % Ненумерованный section, но добавленный в оглавление
  \newcommand\sectiontoc[1]{\section*{#1}\addcontentsline{toc}{section}{#1}}

  \newcommand{\underscore}[1]{\hbox to#1{\hrulefill}}
  \newcommand{\todo}[1]{\textbf{\textcolor{red}{TODO: #1}}}
  \newcommand{\note}[1]{\textit{Примечание: #1}}
  \newcommand{\eng}[1]{{\English #1}}

  % обёртка с моими настройками поверх figure:
  % \begin{myfigure}{подпись}{fig:label} ... \end{myfigure}
  \newenvironment{myfigure}[2]%
    {\pushQED{\caption{#1} \label{#2}} % push caption & label
     \begin{figure}[!htb]\centering } %
    {  \popQED % pop caption & label
     \end{figure}}

  % вставка картинки: \figure[params]{подпись}{file}
  % создаёт label вида fig:file
  \newcommand{\includefigure}[3][]{
    \begin{myfigure}{#2}{fig:#3}
      \includegraphics[#1]{#3}
    \end{myfigure}
  }

  % вставка subfigure внутри myfigure:
  % \subfigure[params]{подпись}{file}
  \newcommand{\subfigure}[3][]{
    \subfloat[#2]{\label{fig:#3}\includegraphics[#1]{#3}}
  }

  \renewcommand{\le}{\leqslant} % <= с наклонной нижней перекладиной
  \renewcommand{\ge}{\geqslant} % >= с наклонной нижней перекладиной

  \linespread{1.3}

  % русские буквы для списков и частей рисунка
  \renewcommand{\theenumii}{(\asbuk{enumii})}
  \renewcommand{\labelenumii}{\asbuk{enumii})}
  \renewcommand{\thesubfigure}{\asbuk{subfigure}}

  % % разделы с новой страницы
  % \let\oldsection\section
  % \renewcommand{\section}{\newpage\oldsection}

  \setcounter{tocdepth}{3} % глубина оглавления

  \hyphenation{англ} % убрать перенос в этом сокращении

  % алиас и настройки для numprint
  \newcommand{\num}[1]{\numprint{#1}}
  \npthousandsep{\,}
  \npthousandthpartsep{}
  \npdecimalsign{,}

  \newcommand{\checkdate}[3]{({\Russian дата обращения: \formatdate{#1}{#2}{#3}})}

  \newcommand{\thetitle}{ИНФОРМАЦИОННАЯ СИСТЕМА ОБРАБОТКИ ПРЕЦИЗИОННЫХ СИГНАЛОВ}
  \newcommand{\theauthor}{Новиков И.А.}
  \newcommand{\theinstitute}{ФГБОУ ВПО «Кубанский государственный университет»}

  \author{\theauthor, \theinstitute}
  \title{\thetitle}

  \hypersetup{
    pdfinfo={
      Title = {\thetitle},
      Author = {\theauthor, КубГУ},
      Subject = {}
    }
  }

\begin{document}

  \begin{center}
    \textbf{\thetitle}

    \theauthor

    \theinstitute, г. Краснодар
  \end{center}

  \sectiontoc{Введение}\label{sec:intro}

  В процессе производственной и познавательной деятельности возникает множество практических и
  теоретических задач, для решения которых необходимо располагать количественной информацией о том
  или ином свойстве объекта. Основным способом получения информации является процесс измерения.
  Поэтому задача повышения эффективности и достоверности процессов измерения и обработки информации
  является актуальной.

  Автоматизация эксперимента подразумевает использование ЭВМ, что позволяет
  собирать, хранить и обрабатывать большое количество информации, управлять экспериментом в процессе
  его проведения, обслуживать одновременно несколько установок \cite{petronevich-automation}.
  Чтобы применение ЭВМ действительно способствовало повышению эффективности измерений, к ЭВМ
  предъявляется требование работы в режиме реального времени
  \cite{tessier-reconfigurable}. С~другой
  стороны, повышение точности по времени связано с увеличением частоты дискретизации, и, как
  следствие, количества обрабатываемых в единицу времени отсчётов данных. Это вынуждает либо искать
  компромисс между реальным временем и высокой пропускной способностью, либо повышать быстродействие
  ЭВМ, что включает в себя как использование более производительного аппаратного обеспечения, так и
  оптимизацию программного обеспечения.

  Помимо получения измерительных данных, ЭВМ также решает задачи их обработки, сохранения и
  визуального представления. Из того, что эти процессы должны протекать одновременно и независимо,
  вытекает ещё одна особенность программного обеспечения для обработки измерительных данных~---
  необходимость использования параллельных вычислений. Это требование, как и требование
  быстродействия, влияет и на выбор аппаратного обеспечения, и на реализацию программного обеспечения.

  Системы автоматизированной обработки измерительных данных находят применение в различных отраслях,
  в том числе, в геофизике (регистрация землетрясений и поиск полезных ископаемых), радиофизике,
  технологическом контроле, специализированной измерительной технике.

  Целью данной работы является разработка информационной системы обработки измерительных данных,
  то~есть специализированного программного обеспечения, предназначенного для эффективной
  автоматизации процесса обработки прецизионных данных. Для достижения этой цели ставятся задачи,
  которые подробно описаны далее, в подразделе~\ref{ssec:task}.

  \section{Описание предметной области}

  Общие принципы и требования, предъявляемые к автоматизации эксперимента \cite{vinogradov-discrete, kurochkin-kamak}: % TODO 1981 - старый источник!
  \begin{itemize}
    \item повышенные требования к быстродействию автоматизированных систем, поскольку такие системы
      предназначены для быстрого получения и анализа данных и оперативного принятия решений;
    \item высокая надёжность автоматизированных систем, возможность длительной безотказной работы;
    \item простота эксплуатации автоматизированных систем и использование унифицированных блоков;
    \item гибкость автоматизированных систем, допускающая изменение её состава и структуры в процессе работы;
    \item возможность коллективного обслуживания различных установок;
    \item наличие диалогового режим работы, когда осуществляется непосредственная связь человека с системой;
    \item простая и быстрая система контроля.
  \end{itemize}

  \subsection{Применение микроконтроллеров}

  Перспективным решением построения каналов измерения является схема построения, в которой
  используется промежуточная микроЭВМ, в качестве которой могут быть применены как
  специализированные цифровые устройства, так и микроконтроллеры или цифровые сигнальные процессоры.
  Использование микроконтроллеров в измерительном процессе позволяет не только решить задачу
  синхронизации оцифровки сигнала по времени одновременно во всех каналах измерения, но также
  существенно расширить функциональные возможности измерительной аппаратуры за счёт организации
  таких операций, как калибровка, поверка, минимизация дестабилизирующих факторов, контроль и
  управление измерительными сенсорами, первичное преобразование и обработка, а также существенно
  упростить задачу передачи информации по каналам связи \cite{klaassen-methods}.

  Процесс автоматизированного измерения при этом происходит следующим образом.
  Данные от датчиков физических величин поступают на масштабные преобразователи измерительных
  каналов. %Использование при этом микроконтроллера с обратной связью позволяет организовывать
  %процесс прецизионного измерения входных величин за счёт разделения информации по каналам
  %измерения с разным коэффициентом усиления и её сложением в микроконтроллере, в этом случае часть
  %измерительных каналов работает с большим сигналом, а часть настроена на малый сигнал, что
  %позволяет измерять аналоговые сигналы в большом диапазоне значений, при этом на выходе
  %микроконтроллера мы получаем непрерывный сигнал во всем диапазоне значений физических датчиков.
  %Устройства фильтрации также управляются от микроконтроллера, что позволяет не только выполнять
  %стандартные операции фильтрации сигнала, но и, применяя специализированные алгоритмы обработки
  %информации, оперативно реагировать на изменяющиеся помехи и оперативно адаптировать параметры
  %фильтрации, что существенно расширяет возможности всей автоматизированной системы и увеличивает
  %общую помехоустойчивость процесса измерения сигналов.
  Отфильтрованные устройствами фильтрации данные затем поступают на аналого-цифровые преобразователи, которые, в свою очередь,
  управляются по линии синхронизации. Это позволяет решить задачу синхронной дискретизации сигналов
  по времени. %, при этом точность синхронизации будет определяться, в первую очередь, точностью
  %устройства синхронизации, в качестве которого могут использоваться как классические источники
  %высокоточных синхроимпульсов, так и генераторы синхроимпульсов, построенные с использованием
  %глобальной системы позиционирования (GPS).
  Информация, поступающая на входы микроконтроллера в цифровом виде от аналого-цифровых
  преобразователей может как передаваться далее по линиями связи в основную ЭВМ, так и
  накапливаться во временном оперативном запоминающем устройстве. %Это позволяет, с одной стороны,
  %применять специализированные алгоритмы обработки информации с целью оперативного управления
  %масштабными преобразователями и устройствами фильтрации, а с другой~--- за счёт применения
  %высокоскоростных параллельных ОЗУ накапливать данные со скоростью превосходящей возможности
  %каналов связи. Такое применение гарантирует, что в заданный промежуток времени весь поток
  %поступающей информации будет обработан и в дальнейшем передан в ЭВМ для анализа и хранения.

  %Применение микроконтроллеров в подготовке информации для передачи по каналам связи, позволяет
  %более полно использовать каждый физический канал связи, что в общем случае позволяет как добиться
  %большей скорости передачи информации, так и существенно расширить спектр применяемых видов каналов
  %связи: от радиоканала до специализированной вычислительной сети. Применение специализированных
  %алгоритмов в подготовке данных позволяет добиться гарантированного прохождения информации по
  %каналам связи без потерь, а применение алгоритмов шифрования и распределения доступа позволяет
  %большему количеству исследователей одновременно участвовать в физических экспериментах, при этом
  %гарантируется надёжность управления всей системой.

  \subsection{Устройство синхронизации}\label{ssec:sync}

  Точная синхронизация по времени процедур обработки
  сигналов возможна за счёт применения вычислительных процессоров и специализированных
  инструментальных средств. Решение этой задачи позволяет синхронизировать отдельные процессы
  обработки информации по времени, что является существенным фактором возможности применения
  цифровой обработки сигналов в специализированных областях радиофизики и геофизики \cite{rathore-digital}.

  Анализ возможных путей решения задачи синхронизации показал, что решить эту задачу возможно как
  классическими источниками высокоточных синхроимпульсов, так и генераторами синхроимпульсов,
  построенные с использованием глобальной системы позиционирования (GPS система). Использование GPS
  системы позволяет добиться точности межканальной синхронизации не менее 100 нс \cite{lombardi-gps}, причём в случае
  распределённого в пространстве объекта измерения нет необходимости направлять синхроимпульсы из
  единого устройства синхронизации, а можно использовать с каждым территориальным измерительным
  каналом своё устройство синхронизации, причём точность синхронизации при этом не изменяется,
  одновременно отсутствует погрешность обусловленная прохождением синхроимпульсов по длинным
  распределённым линиям.

  \subsection{Существующие программные решения}\label{ssec:existing}

  При решении задачи цифровой обработки сигналов программная часть может реализоваться отдельно под
  каждую конкретную задачу, как, например, в работе \cite{bak-autometry}.
  Тем не менее, существуют различные универсальные программные решения, которые можно разбить на две
  основных группы:
  \begin{enumerate}
    \item SCADA-системы (англ. Supervisory Control And Data Acquisition — диспетчерское управление и
      сбор данных), программные пакеты, предназначенные для разработки или обеспечения работы в
      реальном времени систем сбора, обработки, отображения и архивирования информации об объекте
      мониторинга или управления. \cite{boyer-scada}
      Примеры таких систем: InTouch, TraceMode и др.
    \item Платформы графического программирования, такие как LabVIEW \cite{lavrov-labview}, MatLab/Simulink и др.
  \end{enumerate}

  Главными преимуществами SCADA-систем являются их высокая производительность и надёжность,
  определяющиеся областью применения этих систем для контроля за крупными промышленными объектами.
  Это обуславливает и их главные недостатки: высокую стоимость и громоздкость, а~также, в отдельных
  случаях, необходимость использования специального оборудования.

  Платформы второй группы лишены этого недостатка. Однако в отличие от SCADA-систем, они не
  представляют из себя готовые решения, будучи лишь мощными средами для разработки таковых.
  И хотя наличие специализированных инструментов и готовых блоков делает разработку существенно
  проще по сравнению с языками программирования общего назначения, такие системы лучше всего
  подходят для прототипирования и демонстрационных целей, так как имеют худшее быстродействие по
  сравнению с первой группой из-за использования интерпретаторов.

  \subsection{Постановка задачи}\label{ssec:task}

  Как уже было сказано выше в~разделе \nameref{sec:intro},
  целью работы является повышение эффективности и достоверности процессов автоматизированной
  обработки измерительных сигналов. С этой целью разрабатывается универсальная информационная
  система обработки прецизионных сигналов,
  специальное программное обеспечение,
  являющееся компромиссом между двумя группами имеющихся решений, описанных выше в
  подразделе~\ref{ssec:existing}, то есть готовой к интеграции, гибкой системой, но не такой
  громоздкой и сложной, как SCADA-системы.

  К системе предъявляются следующие требования.

  Система должна быть готовой к использованию программой, не требующей модификации программного кода
  для применения к реальной измерительной системе. В частности, система должна реализовывать
  получение данных по протоколу RS-232\cite{sweet-serial} от микроконтроллера, выполняющего оцифровку.

  Необходимо, чтобы быстродействие системы было достаточным для обработки поступающих цифровых
  сигналов в реальном времени. В частности, система должна обрабатывать в секунду не менее 1000
  отсчётов, поступающих на вход.

  Для обеспечения точности привязки по времени система должна осуществлять синхронизацию времени с
  использованием системы GPS, что было объяснено выше, в подразделе~\ref{ssec:sync}.

  Система должна иметь два режима работы: диалоговый, в~котором пользователь может управлять
  программой с помощью графического интерфейса и наблюдать изменения обрабатываемых данных,
  визуализируемых в реальном времени, и фоновый режим, в~котором графический интерфейс отключается
  для экономии ресурсов.

  Наконец, для эффективного использования вычислительной мощности многоядерных центральных
  процессоров, получающих в наши дни всё большее распространение~\cite{steam-hardware}, нужно, чтобы
  вычисления могли выполняться параллельно.

  Для достижения цели ставятся следующие задачи:
  \begin{enumerate}
    \item Реализация процесса получения данных.
    \item Разработка высокопроизводительного модуля обработки цифровых данных.
    \item Реализация процесса синхронизации.
    \item Разработка подсистемы визуализации.
    \item Разработка модуля, позволяющего визуализировать ранее накопленные данные.
  \end{enumerate}

  В следующем разделе будет описан способ выполнения этих задач.

  \section{Реализация системы обработки сигналов}\label{sec:impl}

  Система реализуется в~виде набора программ, написанных на языке C++. Выбор языка обоснован
  компромиссом между быстродействием и удобством разработки и интеграции, которые обеспечивает
  использование объектно-ориентированной парадигмы. Система состоит из следующих программ.

  \begin{enumerate}
    \item Основная программа, реализующая получение, обработку, сохранение и визуализацию данных
      (далее~--- основная программа).
    \item Программа статистической обработки накопленных данных, используемая для визуализации
      файлов, сохранённых основной программой (далее~--- программа обработки накопленных данных).
    \item Вспомогательная программа для генерации тестовых данных, описанная в подразделе~\ref{ssec:testing}
      (далее~--- генератор данных).
  \end{enumerate}

  В последующих подразделах будет описана архитектура этих программ и технологии, используемые при их
  разработке. В соответствии с выбором объектно-ориентированного языка программирования, архитектура будет описываться в
  терминах объектно-ориентированного проектирования (англ. \eng{Object-Oriented Analysis and Design,
  OOAD}) \cite{booch-ooad}.

  \subsection{Архитектура системы}

  Основная программа и программа обработки накопленных данных являются независимыми исполняемыми
  модулями. Взаимодействие между программами происходит через поддержку общего формата файлов, в
  качестве которого используется \eng{Comma Separated Values} (CSV, RFC-4180 \cite{rfc4180}).
  Основная программа в ходе работы записывает файлы в~этом формате, которые затем могут быть прочитаны
  программой обработки накопленных данных. В остальном программы независимы и могут использоваться
  отдельно друг от друга.

  Далее будет рассмотрена архитектура каждой из этих программ.

  \paragraph{Основная программа.}
  Функциональность основной программы чётко разделяется на три части: получение данных, их
  преобразование и визуализация. Поэтому наиболее подходящей архитектурой становится
  <<Модель~--- Представление~--- Контроллер>> (англ. \eng{MVC: Model~--- View~--- Controller})
  \cite{gamma-patterns}, при этом получение данных соответствует уровню <<Модель>>, операции с
  данными~--- уровню <<Контроллер>> и, соответственно, визуализация~--- уровню <<Представление>>.
  %Различают две модификации этой архитектуры, отличающиеся ролью уровня <<Модель>>: так называемые
  %<<Пассивная модель>> и <<Активная модель>>. В~первом случае модель представляет собой лишь
  %источник данных и никак не может воздействовать на представления и контроллер, вместо этого
  %контроллер отслеживает её изменения и он же отвечает за обновление представления по необходимости.
  Поскольку в~данной программе для
  достижения реального времени требуется как можно более быстрая реакция на поступающие данные, то
  при реализации этой программы используется вариант архитектуры MVC с активной моделью.
  В~этом случае модель сама оповещает представления о том, что в ней произошли изменения, а
  заинтересованные представления подписываются на эти сообщения.

  Отделение модели и представления от логики приложения (контроллера) позволяет иметь различные
  взаимозаменяемые реализации как модели, так и представления: в~случае модели это разные протоколы
  обмена данными, в~случае представления~--- различные виды интерфейса: графический, текстовый, и~т.\,д.
  Кроме того, упрощается реализация параллельных вычислений: каждый из трёх уровней может работать
  в~отдельном потоке исполнения.

  \paragraph{Программа обработки накопленных данных.}
  В~программе обработки накопленных данных, как и в основной программе, описанной выше,
  независимость визуализации от обработки данных также делает уместным
  применение архитектуры MVC. Но в~отличие от основной программы, здесь данные не приходят извне, а
  считываются из файла и в~дальнейшем при необходимости редактируются пользователем, то~есть и их
  изменение изнутри, а не извне программы. Всё это приводит к тому, что здесь можно использовать
  более простой вариант архитектуры MVC с <<пассивной>> моделью. Недостатком этого варианта является
  то, что при этом значительная часть программного кода сосредоточена в контроллере, а модель и
  представление тривиальны. Впрочем, для данной программы этот недостаток не является критическим,
  так как её функциональность не настолько велика, чтобы класс-контроллер оказался перегружен.

  \subsection{Используемые технологии}\label{ssec:techno}

  %После того, как сделан выбор языка программирования для реализации системы (в данном случае это
  %C++, как было указано в разделе \ref{sec:impl}), необходимо определиться с набором используемых
  %программных библиотек. В~случае, если подходящие библиотеки для данного языка имеются, и являются тщательно
  %протестированными, то их использование более предпочтительно, чем самостоятельная разработка той
  %же функциональности. Если некий набор библиотек реализует комплексный подход к~решению
  %определённого круга задач, то такие библиотеки могут объединяться в общую систему, существенно
  %определяющую структуру использующего их приложения. В~этом случае говорят уже не об отдельных
  %библиотеках, а о~так называемом программном каркасе, или фреймфорке (англ. \eng{framework}).

  %Далее описан выбор базового MVC-фреймворка, а так же библиотек для реализации
  %конкретных особенностей уровней модели и представления.

  %\paragraph{Базовый MVC-фреймворк.}
  Поскольку MVC выбрана в~качестве основной архитектуры для обеих программ, то особенно важным
  становится выбор библиотеки или фреймворка, отвечающих за реализацию этой модели в программном
  коде. Выбранный фреймворк будет влиять и накладывать свои ограничения и на
  структуру разрабатываемой программы. Именно поэтому особого внимания заслуживают свободные и
  кроссплатформенные библиотеки и фреймворки: их использование не требует дополнительных затрат на
  лицензию, открытый свободный код даёт больший контроль над возможностями библиотеки \cite{open-source},
  а кроссплатформенность исключает проблемы, которые могут впоследствии возникнуть при необходимости
  переноса программы на другую платформу.

  Среди имеющегося в~наши дни свободного кросплатформенных программного обеспечения для разработки
  графического интерфейса на языке C++ наиболее распространены библиотека GTK+ (сокращение от англ.
  \eng{GIMP ToolKit}) \cite{gtk} и фреймворк Qt (произносится <<кьют>>) \cite{qt}. В данной работе
  выбор сделан в~пользу Qt, поскольку этот фреймворк обладает следующими преимуществами: расширение
  языка C++ при помощи метакомпилятора, мощная событийно-ориентированная модель <<сигналов>> и
  <<слотов>>, широкие возможности графического интерфейса с возможностью расширения, простота
  локализации на новый язык.
  %\begin{enumerate}
  %  \item Расширение языка C++, реализуемое с помощью \eng{Meta Object Compiler (MOC)}, который
  %    транслирует расширенный диалект C++ в стандартный C++.
  %  \item Мощная модель так называемых <<сигналов>> и <<слотов>>, позволяющая использовать
  %    событийно-ориентированное программирование, причём MOC обеспечивает лаконичный синтаксис их определения.
  %  \item Широкие возможности графического интерфейса с возможностью расширения стандартного набора
  %    элементов пользовательскими либо элементами из сторонних библиотек (рис.~\ref{fig:qt-creator}).
  %  \item Простота локализации приложения на новый язык благодаря инструменту \eng{Linguist},
  %    входящему в состав Qt.
  %\end{enumerate}
  %В данной работе используется последняя на момент её написания версия фрейворка Qt 5.0.2.

  %\includefigure[width=0.7\linewidth]{среда разработки Qt Creator в режиме визуального редактора.
  %  Видно большое разнообразие графических элементов, представление логической иерархии в виде
  %  дерева, а также редактор меню и сигналов-слотов.}{qt-creator}


  %С учётом выбора Qt в качестве базового фреймворка, необходимо выбрать совместимые с ним библиотеки
  %для важных функций основной программы: получения данных по протоколу RS-232 через последовательный
  %порт и визуализации полученных данных в виде графиков. Выбор этих библиотек будет описан далее.

  %\paragraph{Библиотека для работы с последовательным портом.}
  На момент написания работы в широком доступе имелись две сторонних библиотеки для работы с
  последовательным портом средствами Qt: QtSerialPort \cite{qtserialport} и QextSerialPort \cite{qextserialport}.
  Первая, QtSerialPort, будучи изначально сторонней библиотекой, была принята разработчиками Qt и
  войдёт в его состав, что свидетельствует о качестве этой библиотеки. Однако вторая
  библиотека, QextSerialPort, имеет некоторые серьёзные преимущества перед ней: это давно и активно
  развивающийся проект, то есть он более надёжен; поддерживает широкий набор версий Qt от Qt2 до Qt5
  что исключает проблемы обратной совместимости; наконец, не имеет внешних зависимостей (в то время
  как для работы QtSerialPort необходим Perl). Всё это определяет выбор в~пользу QextSerialPort.
  %В данной работе используется последняя стабильная версия на момент написания~--- QextSerialPort 1.2.

  %\paragraph{Библиотека для построения графиков.}
  Для построения графиков в~графическом интерфейсе Qt имеются две основных альтернативы: QWT (англ.
  \eng{Qt Widgets for Technical Applications}) \cite{qwt} и QCustomPlot \cite{qcustomplot}.  Обе эти
  библиотеки беспрепятственно интегрируются с Qt-приложением и позволяют строить графики разных
  типов. При этом однозначными преимуществами QWT являются более широкий набор функций и
  возможностей конфигурации, подробная документация и более активная разработка и исправление
  ошибок: последняя версия на момент написания работы, 6.1.0~--- от 30 мая 2013 (против 9 июня 2012 в
  случае QCustomPlot). Поэтому в~данной работе используется QWT, а~именно, версия 6.1.0.

  \subsection{Тестирование системы}\label{ssec:testing}

  Для тестирования системы была разработана вспомогательная тестовая программа~--- генератор данных,
  во время тестирования заменяющий собой реальный измерительный прибор: в него загружаются данные из
  файла, а он передаёт их по последовательному порту, преобразуя к~тому виду, который они имели бы,
  если бы приходили с~реального прибора после оцифровки микроконтроллером.

  Генератор данных разрабатывался с~использованием тех же средств, что и основная программа:
  фреймворка Qt и библиотеки QextSerialPort, которые были описаны в подразделе~\ref{ssec:techno}.
  %Внешний вид этой программы показан на рис.~\ref{fig:conserial}.

  %\includefigure[width=0.4\textwidth]{Тестовая программа~--- генератор данных}{conserial}

  %Процесс отладки во время разработки происходит следующим образом. На персональный компьютер,
  %используемом при разработке, устанавливается эмулятор COM-порта, создающий два виртуальных порта
  %ведущих себя так, как если бы они были соединены друг с другом кабелем. При помощи генератора
  %данных данные передаются основной программе через этот виртуальный порт. Тем самым проверяется
  %корректность реализации взаимодействия с последовательным портом и логики приложения, хотя и
  %генератор данных, и основная программа физически находятся на одном компьютере.

  %Для окончательного тестирования применяется реальное соединение.
  Тестирование основной программы с использованием генератора данных происходит следующим образом.
  Два персональных компьютера
  соединяются нуль-модемным кабелем \cite{null-modem}, на одном из них запускается основная
  программа, на другом~--- генератор данных. После установления соединения в~генератор данных
  загружаются данные из файла, и он отправляет их через это соединение на другой компьютер, где их
  получает и отображает основная программа. В качестве данных использовались реальные данные с
  сейсмических датчиков, записанные на испытательном стенде для сейсмографов. Проведённое
  тестирование подтвердило корректность работы основной программы.

  \subsection{Результаты}

  Основная программа разработана и протестирована на реальных сигналах. Внешний вид программы
  представлен на рис.~\ref{fig:main}.
  \includefigure[width=0.7\textwidth]{основная программа}{main}

  %Принцип работы пользователя с данной программой следующий. Для её работы необходимо подключить
  %микроконтроллер, выполняющий оцифровку данных, при помощи последовательного порта к персональному
  %компьютеру, на котором запускается основная программа. После запуска программы указывается порт,
  %по которому выполнено подключение, после чего нажатием кнопки <<Подключить>> инициализируется
  %процесс соединения с микроконтроллером и обмена командами. При успешном его завершении программа
  %переходит в режим готовности к приёму данных, и становится доступной кнопка <<Запуск>>. При её нажатии
  %программа отправляет микроконтроллеру команду, разрешающую передачу данных и начинает
  %регистрировать получаемые данные. Зарегистрированные данные отображаются в~табличном и
  %в~графическом виде: в~виде графика зависимости уровня сигнала от времени. По мере работы также
  %отображается время начала приёма, прошедшее время и количество полученных данных. Приём данных
  %прекращается кнопкой <<Стоп>>, также имеется возможность закрыть соединение с микроконтроллером
  %нажатием кнопки <<Отключить>>, после чего программа переходит в изначальное состояние и работу с
  %ней можно начинать с начала.

  Программа обработки накопленных данных разработана, также протестирована на данных реального
  сигнала. Было выяснено, что программа может открывать файлы объёмом вплоть до \num{100000} записей
  и рассчитывать для них статистику без серьёзного замедления в работе.% Внешний вид программы
  %показан на рис.~\ref{fig:stat}.
  %\includefigure[width=0.6\textwidth]{программа обработки накопленных данных}{stat}

  %Помимо использования вместе с основной программой, данная программа может найти применение при
  %исследовании физических процессов, связанных с измерением физических параметров и эффективно
  %проводить качественный и количественный анализ экспериментальных данных с оперативным изменением
  %процесса измерения. В программе реализованы следующие функции: ввод экспериментальных данных с
  %клавиатуры и из файла, редактирование введённых данных, расчёт среднего значения,
  %среднеквадратичного отклонения, дисперсии и моментов, построение настраиваемой гистограммы и кривой
  %данных, генерация отчёта с возможностью сохранения его в файл в формате HTML, а также сохранения
  %графиков в виде изображения PNG. При открытии или создании нового файла программа рассчитывает все
  %перечисленные величины, строит кривую данных и гистограмму. В дальнейшем, при каждом
  %изменении данных пользователем эти величины пересчитываются, графики обновляются. Таким образом,
  %имеется возможность вводить и обновлять данные об эксперименте и уже по мере ввода видеть
  %изменение статистических значений, а по завершении ввода получить готовый ответ о среднем значении
  %и его точности в виде $x \pm \Delta x$.

  Программа обработки накопленных данных была зарегистрирована в Реестре программ для ЭВМ в феврале 2013
  года. (Свидетельство о гос. регистрации программы №~2013612036, Программа ProcessData)
  %%\section{Дальнейшая работа}

  \sectiontoc{Заключение}

  В данной работе ставилась цель разработать программную систему обработки прецизионных сигналов,
  работающую в реальном времени и обеспечивающую помимо обработки точную синхронизацию по времени и
  визуализацию обрабатываемых данных. Эта цель была достигнута, и были удовлетворены поставленные в
  подразделе~\ref{ssec:task} требования, включая высокое быстродействие и гибкость, а~также применение
  параллельных вычислений.

  Проведено тестирование с использованием вместо реального измерительного прибора вспомогательной
  программы, передающей через последовательный порт тестовые данные. В качестве тестовых данных
  использовались реальные данные с сейсмических датчиков. Тестирование показало пригодность системы
  для реальных измерений.

  Была также разработана и зарегистрирована программа для статистической обработки накопленных
  данных, которая может быть использоваться и как часть системы, и отдельно от неё, например, при
  проведении лабораторных работ по физике.

  Также были исследованы возможности дальнейшего развития и улучшения системы. В частности,
  планируется расширение набора поддерживаемых протоколов для получения данных и добавление новых
  операций обработки данных, а также реализация синхронизации с~помощью GPS.

  %Работа была представлена на заседании Студенческого научного общества физико-технического факультета
  %КубГУ в апреле 2013 года и награждена дипломом третьей степени.

  \begin{flushleft}
    \bibliographystyle{../../biblio/ugost2003} % Или использовать ugost2008 для нового ГОСТа
    \bibliography{../../biblio/my}
  \end{flushleft}
\end{document}
