\documentclass[a4paper, 14pt, titlepage]{extarticle}
  \usepackage{cmap}
  \usepackage[hidelinks,pdftex,unicode]{hyperref}
  \usepackage{mathtext} % для кириллицы в формулах
  \usepackage[T2A]{fontenc}
  \usepackage[utf8]{inputenc}
  \usepackage[english,russian]{babel}
  \usepackage{indentfirst}
  \usepackage{cite}
  \usepackage{amsmath} % для \eqref
  \usepackage{amssymb} % для \leqslant
  \usepackage{amsthm} % для \pushQED
  \usepackage{color} % пока только для TODO:
  \usepackage[pdftex]{graphicx}
  \usepackage{subfig}
  \usepackage{numprint}
  \usepackage[left=30mm,right=15mm,top=20mm,bottom=20mm,bindingoffset=0cm]{geometry}
  \usepackage{datetime}
  \graphicspath{{../img/}{../../img/}}
  \frenchspacing

  \DeclareSymbolFont{T2Aletters}{T2A}{cmr}{m}{it} % кириллица в формулах курсивом

  \addto\captionsrussian{
    \renewcommand\contentsname{Содержание}
    % перекрываю \refname, чтобы список литературы сам добавлял себя в оглавление
    \let\oldrefname\refname
    \renewcommand\refname{\addcontentsline{toc}{section}{\oldrefname}\oldrefname}
  }

  \newcommand{\underscore}[1]{\hbox to#1{\hrulefill}}
  \newcommand{\todo}[1]{\textbf{\textcolor{red}{TODO: #1}}}
  \newcommand{\note}[1]{\textit{Примечание: #1}}
  \newcommand{\eng}[1]{{\English #1}}

  % обёртка с моими настройками поверх figure:
  % \begin{myfigure}{подпись}{fig:label} ... \end{myfigure}
  \newenvironment{myfigure}[2]%
    {\pushQED{\caption{#1} \label{#2}} % push caption & label
     \begin{figure}[!htb]\centering } %
    {  \popQED % pop caption & label
     \end{figure}}

  % вставка картинки: \figure[params]{подпись}{file}
  % создаёт label вида fig:file
  \newcommand{\includefigure}[3][]{
    \begin{myfigure}{#2}{fig:#3}
      \includegraphics[#1]{#3}
    \end{myfigure}
  }

  % вставка subfigure внутри myfigure:
  % \subfigure[params]{подпись}{file}
  \newcommand{\subfigure}[3][]{
    \subfloat[#2]{\label{fig:#3}\includegraphics[#1]{#3}}
  }

  \renewcommand{\le}{\leqslant} % <= с наклонной нижней перекладиной
  \renewcommand{\ge}{\geqslant} % >= с наклонной нижней перекладиной

  \linespread{1.3}

  % русские буквы для списков и частей рисунка
  \renewcommand{\theenumii}{(\asbuk{enumii})}
  \renewcommand{\labelenumii}{\asbuk{enumii})}
  \renewcommand{\thesubfigure}{\asbuk{subfigure}}

  % TODO вернуть секции с новой страницы по необходимости
  % \let\oldsection\section
  % \renewcommand{\section}{\newpage\oldsection}

  \setcounter{tocdepth}{3} % глубина оглавления

  \hyphenation{англ} % убрать перенос в этом сокращении

  % алиас и настройки для numprint
  \newcommand{\num}[1]{\numprint{#1}}
  \npthousandsep{\,}
  \npthousandthpartsep{}
  \npdecimalsign{,}

  \newcommand{\checkdate}[3]{({\Russian дата обращения: \formatdate{#1}{#2}{#3}})}

  \author{И.\,А.\,Новиков, кафедра физики и информационных систем КубГУ}
  \title{Информационная система обработки экспериментальных данных}

\begin{document}

%----------------------- титульный лист ------------------------

  \thispagestyle{empty}
  \begin {center}
  МИНИСТЕРСТВО ОБРАЗОВАНИЯ И НАУКИ РОССИЙСКОЙ ФЕДЕРАЦИИ\\
  Федеральное государственное бюджетное образовательное учреждение\\
  высшего профессионального образования\\
  «КУБАНСКИЙ ГОСУДАРСТВЕННЫЙ УНИВЕРСИТЕТ»\\
  (ФГБОУ ВПО «КубГУ»)

  Физико-технический факультет

  \vspace {1cm}

  Кафедра физики и информационных систем

  \vspace {3.5cm}

  \textbf{КУРСОВАЯ РАБОТА}

  \vspace {0.5cm}

  \textbf{ИНФОРМАЦИОННАЯ СИСТЕМА ОБРАБОТКИ ЭКСПЕРИМЕНТАЛЬНЫХ ДАННЫХ}

  \vspace {1.5cm}

  \begin{flushleft}
    Работу выполнил \underscore{5cm} (Новиков Иван Александрович)\\
    Направление магистерской подготовки 011200.68 Физика

    Руководитель магистерской программы <<Информационные процессы и системы>>\\
    профессор, д-р физ.-мат. наук \underscore{5cm} (Н.\,М.\,Богатов)

    Научный руководитель\\
    доцент, канд. физ.-мат. наук \underscore{5cm} (Л.\,Р.\,Григорьян)

    Нормоконтролер\\
    должность, уч. степень \underscore{5cm} (И.\,О.\,Фамилия)
  \end{flushleft}

  \vspace {2cm}

  Краснодар~--- 2012~год
  \end {center}

%------------------------- содержание -------------------------

  \tableofcontents
  % \newpage

%-------------------------- введение --------------------------
  \section{Введение}

  В процессе производственной и познавательной деятельности возникает множество практических и
  теоретических задач, для решения которых необходимо располагать количественной информацией о том
  или ином свойстве объекта. Основным способом получения информации является процесс измерения.
  Поэтому, задача повышения эффективности и достоверности процессов измерения и обработки информации
  является актуальной.

  \section{Многоканальные измерительные системы}

  Существует достаточно широкий набор средств, которые позволяют создавать сложные системы
  измерения, контроля и мониторинга физических процессов. Автоматизация эксперимента - комплекс
  средств и методов для ускорения сбора и обработки экспериментальный данных, интенсификации
  использования экспериментальных установок, повышения эффективности работы исследователей.
  Характерной особенностью автоматизации эксперимента  является использование ЭВМ, что позволяет
  собирать, хранить и обрабатывать большое количество информации, управлять экспериментом в процессе
  его проведения, обслуживать одновременно несколько установок \cite{sokolov-auto-measure}.

  \begin{flushleft}
    \bibliographystyle{../../biblio/gost705}
    \bibliography{../../biblio/my}
  \end{flushleft}
\end{document}

