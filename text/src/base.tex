\documentclass[a4paper, 14pt, titlepage]{extarticle}
  \usepackage{cmap}
  \usepackage[hidelinks,pdftex,unicode]{hyperref}
  \usepackage{mathtext} % для кириллицы в формулах
  \usepackage[T2A]{fontenc}
  \usepackage[utf8]{inputenc}
  \usepackage[english,russian]{babel}
  \usepackage{indentfirst}
  \usepackage{cite}
  \usepackage{amsmath} % для \eqref
  \usepackage{amssymb} % для \leqslant
  \usepackage{amsthm} % для \pushQED
  \usepackage[usenames,dvipsnames]{xcolor}
  \usepackage[pdftex]{graphicx}
  \usepackage{subfig}
  \usepackage{numprint}
  \usepackage[left=30mm,right=15mm,top=20mm,bottom=20mm,bindingoffset=0cm]{geometry}
  \usepackage{datetime}
  \usepackage{setspace} % для иных отступов в tikz-картинках
  \usepackage{tikz}
    \usetikzlibrary{positioning,fit,shapes,calc}
  \graphicspath{{../img/}{../../img/}}
  \frenchspacing

  \DeclareSymbolFont{T2Aletters}{T2A}{cmr}{m}{it} % кириллица в формулах курсивом

  \addto\captionsrussian{
    \renewcommand\contentsname{Содержание}
    % перекрываю \refname, чтобы список литературы сам добавлял себя в оглавление
    \let\oldrefname\refname
    \renewcommand\refname{\addcontentsline{toc}{section}{\oldrefname}\oldrefname}
  }

  % Ненумерованный section, но добавленный в оглавление
  \newcommand\sectiontoc[1]{\section*{#1}\addcontentsline{toc}{section}{#1}}

  \newcommand{\underscore}[1]{\hbox to#1{\hrulefill}}
  \newcommand{\todo}[1]{\textbf{\textcolor{red}{TODO: #1}}}
  \newcommand{\note}[1]{\textit{Примечание: #1}}
  \newcommand{\eng}[1]{{\English #1}}

  % обёртка с моими настройками поверх figure:
  % \begin{myfigure}{подпись}{fig:label} ... \end{myfigure}
  \newenvironment{myfigure}[2]%
    {\pushQED{\caption{#1} \label{#2}} % push caption & label
     \begin{figure}[!htb]\centering } %
    {  \popQED % pop caption & label
     \end{figure}}

  % вставка картинки: \figure[params]{подпись}{file}
  % создаёт label вида fig:file
  \newcommand{\includefigure}[3][]{
    \begin{myfigure}{#2}{fig:#3}
      \includegraphics[#1]{#3}
    \end{myfigure}
  }

  % вставка subfigure внутри myfigure:
  % \subfigure[params]{подпись}{file}
  \newcommand{\subfigure}[3][]{
    \subfloat[#2]{\label{fig:#3}\includegraphics[#1]{#3}}
  }

  \renewcommand{\le}{\leqslant} % <= с наклонной нижней перекладиной
  \renewcommand{\ge}{\geqslant} % >= с наклонной нижней перекладиной

  \linespread{1.3}

  % русские буквы для списков и частей рисунка
  \renewcommand{\theenumii}{(\asbuk{enumii})}
  \renewcommand{\labelenumii}{\asbuk{enumii})}
  \renewcommand{\thesubfigure}{\asbuk{subfigure}}

  % разделы с новой страницы
  \let\oldsection\section
  \renewcommand{\section}{\newpage\oldsection}

  \setcounter{tocdepth}{3} % глубина оглавления

  \hyphenation{англ} % убрать перенос в этом сокращении

  % алиас и настройки для numprint
  \newcommand{\num}[1]{\numprint{#1}}
  \npthousandsep{\,}
  \npthousandthpartsep{}
  \npdecimalsign{,}

  \newcommand{\checkdate}[3]{({\Russian дата обращения: \formatdate{#1}{#2}{#3}})}

  \newcommand{\thetitle}{Информационная система обработки прецизионных сигналов}
  \newcommand{\theauthor}{Иван Новиков}

  \author{\theauthor, кафедра физики и информационных систем КубГУ}
  \title{\thetitle}

  \hypersetup{
    pdfinfo={
      Title = {\thetitle},
      Author = {\theauthor},
      Subject = {}
    }
  }

\begin{document}

%----------------------- титульный лист ------------------------

  \thispagestyle{empty}
  \begin {center}
  МИНИСТЕРСТВО ОБРАЗОВАНИЯ И НАУКИ РОССИЙСКОЙ ФЕДЕРАЦИИ\\
  Федеральное государственное бюджетное образовательное учреждение\\
  высшего профессионального образования\\
  «КУБАНСКИЙ ГОСУДАРСТВЕННЫЙ УНИВЕРСИТЕТ»\\
  (ФГБОУ ВПО «КубГУ»)

  Физико-технический факультет

  \vspace {1cm}

  Кафедра физики и информационных систем

  \vspace {3.5cm}

  \textbf{КУРСОВАЯ РАБОТА}

  \vspace {0.5cm}

  \textbf{ \large \scshape \thetitle }

  \vspace {1.5cm}

  \begin{flushleft}
    Работу выполнил \underscore{5cm} (Новиков Иван Александрович)\\
    Направление магистерской подготовки 011200.68 Физика

    Руководитель магистерской программы <<Информационные процессы и системы>>\\
    профессор, д-р физ.-мат. наук \underscore{5cm} (Н.\,М.\,Богатов)

    Научный руководитель\\
    доцент, канд. физ.-мат. наук \underscore{5cm} (Л.\,Р.\,Григорьян)

    Нормоконтролер\\
    доцент, канд. физ.-мат. наук \underscore{5cm} (Л.\,Р.\,Григорьян)
  \end{flushleft}

  \vspace {2cm}

  Краснодар~--- 2012~год
  \end {center}

%------------------------- содержание -------------------------

  \tableofcontents
  % \newpage

%-------------------------- введение --------------------------
  \sectiontoc{Введение}

  В процессе производственной и познавательной деятельности возникает множество практических и
  теоретических задач, для решения которых необходимо располагать количественной информацией о том
  или ином свойстве объекта. Основным способом получения информации является процесс измерения.
  Поэтому задача повышения эффективности и достоверности процессов измерения и обработки информации
  является актуальной.

  Существует достаточно широкий набор средств, которые позволяют создавать сложные системы
  измерения, контроля и мониторинга физических процессов. Автоматизация эксперимента~--- комплекс
  средств и методов для ускорения сбора и обработки экспериментальный данных, интенсификации
  использования экспериментальных установок, повышения эффективности работы исследователей.
  Характерной особенностью автоматизации эксперимента является использование ЭВМ, что позволяет
  собирать, хранить и обрабатывать большое количество информации, управлять экспериментом в процессе
  его проведения, обслуживать одновременно несколько установок \cite{sokolov-auto-measure}.

  Чтобы применение ЭВМ действительно способствовало повышению эффективности измерений, к ЭВМ
  предъявляется важное требование: она должна работать в режиме реального времени
  \cite{tessier-reconfigurable}. В частности, недопустимы задержки отклика, которые могут приводить
  к потерям получаемых данных и некорректной привязке приходящих данных ко времени. С~другой
  стороны, увеличение точности по времени связано с увеличением частоты дискретизации, и, как
  следствие, количества обрабатываемых в единицу времени отсчётов данных. Это вынуждает либо искать
  компромисс между реальным временем и высокой пропускной способностью, либо повышать быстродействие
  ЭВМ, что включает в себя как использование более производительного аппаратного обеспечения, так и
  оптимизацию программного обеспечения.

  Помимо получения измерительных данных, ЭВМ также решает задачи их обработки, сохранения и
  визуального представления. Из того, что эти процессы должны протекать одновременно и независимо,
  вытекает ещё одна особенность программного обеспечения для обработки измерительных данных~---
  необходимость использования параллельных вычислений. Это требование, как и требование
  быстродействия, влияет и на выбор аппаратного обеспечения, и на реализацию программного обеспечения.

  Системы автоматизированной обработки измерительных данных находят применение в различных отраслях,
  в том числе, в геофизике (регистрация землетрясений и поиск полезных ископаемых), радиофизике,
  технологическом контроле, специализированной измерительной технике. % TODO ссылки! + подробнее

  \section{Описание предметной области}

  Общие принципы и требования, предъявляемые к автоматизации эксперимента \cite{vinogradov-discrete, kurochkin-kamak}:
  \begin{itemize}
    \item повышенные требования к быстродействию автоматизированных систем, поскольку такие системы
      предназначены для быстрого получения и анализа данных и оперативного принятия решений;
    \item высокая надёжность автоматизированных систем, возможность длительной безотказной работы;
    \item простота эксплуатации автоматизированных систем и использование унифицированных блоков;
    \item необходимость предварительного планирования исследований и разработка возможных вариантов;
    \item гибкость автоматизированных систем, допускающая изменение её состава и структуры в процессе работы;
    \item возможность коллективного обслуживания различных установок;
    \item наличие диалогового режим работы, когда осуществляется непосредственная связь человека с системой;
    \item простая и быстрая система контроля.
      % TODO это хорошо бы куда-то в другое место:
      %Для контроля системы в целом обычно вводят как
      %качественные, так и количественные критерии, характеризующие работу системы в целом. Таким
      %критерием может быть результат измерения известной величины: если полученные значения находятся
      %в допустимых пределах, то состояние системы считается удовлетворительным.
  \end{itemize}

  К этим общим требованиям могут добавляться требования, определяемые конкретной задачей, например,
  использование определённого протокола передачи данных. Кроме того, сами общие требования, как
  правило, конкретизируются. Так, требования быстродействия формулируются в виде конкретных цифр:
  максимальное время отклика, предельный срок завершения, разброс значений времени отклика.
  % TODO вода! Надо более обоснованно, с ссылками
  % NB: эти три параметра взяты с http://ru.wikipedia.org/wiki/Реальное_время

  В последующих подразделах будут описаны основные особенности построения измерительных систем,
  использование микроконтроллеров в них, виды цифровой обработки сигналов, способы синхронизации и
  существующие решения для программного обеспечения таких систем.

  \subsection{Многоканальные измерительные системы}

  Обозначенные принципы позволяют сформировать структуру автоматизированной системы управления
  экспериментом. Данные об исследуемом объекте от специализированных датчиков физических величин
  поступают на вход каналов измерения, которые выполняют функции фильтрации, масштабирования,
  аналого-цифрового преобразования, коррекции данных, и далее данные по каналам связи поступают в
  ЭВМ, где происходит их основная обработка в соответствии с заданными математическими алгоритмами.
  При этом за процессом обработки следит исследователь с помощью пульта контроля и управления, что
  позволяет в реальном масштабе времени вносить необходимые коррективы, как в алгоритмы обработки,
  так и в алгоритмы управления всей системой измерения. Вся измерительная система охвачена каналом
  обратной связи, что позволяет напрямую управлять каждым из компонентов системы от ЭВМ
  (рис.~\ref{fig:system-structure}) \cite{kuzmichev-automation}.

  \begin{myfigure}{структурная схема автоматизированной системы экспериментальных исследований}{fig:system-structure}
    \begin{tikzpicture}[scale=0.9, transform shape]
      \tikzstyle{arrow}  = [->, thick]
      % item - общий стиль-предок для block, comp, control и user
      \tikzstyle{item}   = [rectangle, rounded corners, text centered, font=\footnotesize]
      \tikzstyle{block}  = [item, draw=blue, fill=cyan!10, text width=2.5cm, minimum height=4em]
      \tikzstyle{comp}   = [item, draw=orange, fill=orange!20!yellow!20,
                            text width=2cm, font=\small, minimum height=8em]
      \tikzstyle{object} = [circle, draw=OliveGreen, fill=green!10,
                            text centered, text width=2.2cm, font=\footnotesize]
                            % minimum height должна совпадать с той, что у block
      \tikzstyle{control}= [item, draw=teal, fill=teal!10,text width=4cm]
      \tikzstyle{user}   = [item, draw=violet, fill=violet!10, inner sep=5mm]
      % Блоки верхнего ряда (b = block)
      \node[object] (b0) {Исследуемый\\объект};
      \node[block]  (b1) [right=0.75cm of b0] {Датчики\\физических\\величин};
      \node[block]  (b2) [right=0.75cm of b1] {Каналы\\измерений};
      \node[block]  (b3) [right=0.75cm of b2] {Канал связи};
      % Большой блок "ЭВМ" (comp = computer)
      \node[comp] (Comp) [right=0.75cm of b3, shift={(0,-1)}] {ЭВМ};
      % Узлы для стрелок под блоками
      \foreach \x in {1,...,3} {
        \node (u\x)[below=of b\x] {};
      }
      % Узел под b0 располагаем с осторожностью: b0 имеет другую высоту
      \node (u0) at (u1-|b0) {};  % ( A -| B ) - точка с y-координатой от A и x-координатой от B
      % Стрелки под блоками
      \foreach \x in {0,...,3} {
        \draw [arrow] (u\x.north) -- (b\x);
      }
      % Нижняя объемлющая часть стрелки
      \draw [-, thick] (b0) -- (u0.north) -- (u0.north-|Comp.west) % ( A -| B ) - точка с y-координатой от A и x-координатой от B
            node[midway, below, font=\footnotesize] (Label) {Канал обратной связи};
      % Стрелки между блоками
      \foreach \from/\to in {b0/b1, b1/b2, b2/b3}
        \draw [arrow] (\from) -- (\to);
      % Горизонтальная стрелка между последним блоком и ЭВМ
      \draw [arrow] (b3) -- (b3-|Comp.west);

      \node[control] (Control) [below=0.5cm of  Label ] {Пульт контроля\\и управления};
      \node[user]    (User)    [below=0.75cm of Control] {Исследователь};

      \draw [<->, very thick, blue!50!black] (Comp) |- (Control);
      \draw [<->, ultra thick, brown!75!violet] (User) -- (Control);

      \draw [<-, very thick, red] (b0.west) -- ++(-0.5cm, 0cm) |- (Control);

    \end{tikzpicture}
  \end{myfigure}

  При практической реализации автоматизированной системы каналы измерения представляют собой
  наиболее сложный узел, так как определяют метрологические характеристики всей системы. Типичная
  схема реализации измерительного канала приведена на рис.~\ref{fig:channel-scheme}.

  \begin{myfigure}{функциональная схема канала измерения}{fig:channel-scheme}
    \begin{tikzpicture}[scale=0.9, transform shape]
      \tikzstyle{block} = [rectangle, draw=blue, rectangle, rounded corners, fill=cyan!10,
                           text centered, text width=3cm, font=\footnotesize, minimum height=4em]
      \tikzstyle{around} = [rectangle, draw=darkgray, loosely dashed, rectangle, rounded corners]
      % Блоки (c = channel)
      \node        (c0) {};
      \node[block] (c1) [right=0.5cm of c0] {Масштабный преобразователь};
      \node[block] (c2) [right=0.7cm of c1] {Устройство фильтрации};
      \node[block] (c3) [right=0.7cm of c2] {Аналогово-цифровой \mbox{преобразователь}}; % mbox чтобы избежать переноса
      \node[block] (c4) [right=0.7cm of c3] {Устройство управления};
      \node        (c5) [right=0.5cm of c4] {};

      \node[above=of current bounding box.center] (title) {Измерительный канал};

      % Стрелки под блоками
      \foreach \x in {1,...,3} {
        \node (u\x)[below=of c\x] {};
        \draw [->, thick] (c4) |- (u\x.north) -- (c\x);
      }

      \foreach \from/\to in {c0/c1, c1/c2, c2/c3, c3/c4, c4/c5}
        \draw [->, thick] (\from) -- (\to);

      \node[fit=(c0)(c3)(c5)(u1)(title),around]{\hspace*{\fill}~}; % hspace - хак, чтобы подавить underfull hbox
    \end{tikzpicture}
  \end{myfigure}

  Данные с датчика физических величин поступают на масштабный преобразователь (например, усилитель),
  назначение которого связанно с преобразованием уровня аналогового сигнала до приемлемого значения.
  Далее аналоговый сигнал поступает на устройство фильтрации, которое в общем случае может состоять
  из разных по назначению фильтров, построенных как по аналоговой схеме, так и по цифровой. После
  фильтрации сигнал поступает на аналого-цифровой преобразователь, выполняющий
  функцию дискретизации аналогового сигнала. Устройство управления канала измерения
  выполняет функцию контроля и управления входящими в канал устройствами. Такое построение позволяет
  достаточно просто реализовать канал измерения, однако не обеспечивает решения всего спектра
  возможных измерительных задач. Действительно, использование раздельного управления в каждом канале
  измерения с одной стороны требует усложнения общего управления всеми каналами измерения (требует
  настройки каждого канала измерения по отдельности), с другой стороны не позволяет точно
  синхронизировать измерения по каналам, и в третьих, в силу отсутствия промежуточных вычислительных
  процессоров, не позволяет всей автоматизированной системе в реальном времени реагировать на
  изменение входной информации \cite{stupin-methods}.

  \subsection{Применение микроконтроллеров}

  Одним из решений указанных проблем является применение оптимальных программно-аппаратных
  алгоритмов обработки информации с точной синхронизацией по времени частотно-временных и
  энергетических процедур обработки сигналов с применением вычислительных процессоров и
  специализированных инструментальных средств.

  Перспективным решением построения каналов измерения является схема построения, в которой
  используется промежуточная микроЭВМ, в качестве которой могут быть применены как
  специализированные цифровые устройства, так и микроконтроллеры или цифровые сигнальные процессоры.
  Использование микроконтроллеров в измерительном процессе позволяет не только решить задачу
  синхронизации оцифровки сигнала по времени одновременно во всех каналах измерения, но также
  существенно расширить функциональные возможности измерительной аппаратуры за счёт организации
  таких операций, как калибровка, поверка, минимизация дестабилизирующих факторов, контроль и
  управление измерительными сенсорами, первичное преобразование и обработка, а также существенно
  упростить задачу передачи информации по каналам связи \cite{klaassen-methods}.

  Наиболее важным аспектом применения микроконтроллеров является возможность организации цифровой
  обработки информации с применением процедуры распараллеливания, что позволяет получить
  существенный выигрыш при применении алгоритмов цифровой фильтрации: существенно сократить
  погрешность квантования измерительного сигнала, повысить помехоустойчивость измерительного канала
  и повысить достоверность в выделении отдельных составляющих сигнала, соответствующих тем или иных
  свойствам исследуемого процесса. Одновременно возможно синхронизировать отдельные процессы
  обработки информации по времени, что является существенным фактором возможности применения
  цифровой обработки сигналов в специализированных областях радиофизики и геофизики
  \cite{rathore-digital}. В то же время, применение цифровых фильтров при обработке измерительной
  информации приводит в общем случае к существенному искажению результатов измерений, и, особенно, к
  рассинхронизации измерительной информации по отдельным измерительным каналам, связанной с
  различной задержкой во времени прохождения сигналов по измерительному тракту, поэтому задача
  синхронной обработки информации становится наиболее важной.

  Функциональная схема каналов измерения с применением микроконтроллера  представлена на
  рис.~\ref{fig:channel-with-microcontroller}.

  \begin{myfigure}{функциональная схема канала измерения с применением микроконтроллера}{fig:channel-with-microcontroller}
    \begin{tikzpicture}[scale=0.85, transform shape]
      \tikzstyle{arrow}  = [->, thick]
      \tikzstyle{cline}  = [thick, Blue] % линия управления: толстый синий
      \tikzstyle{sline}  = [thick, Red] % линия синхронизации толстый синий
      \tikzstyle{item}   = [rectangle, rounded corners, text centered, font=\footnotesize]
      \tikzstyle{block}  = [item, draw=blue, fill=cyan!10, text width=3cm, minimum height=4em]
      \tikzstyle{around} = [item, draw=darkgray, loosely dashed]
      \tikzstyle{title}  = [above, font=\small]
      \tikzstyle{controller} = [item, draw=Blue, thick, fill=blue!10, text width=3cm, minimum height=2.5cm]
      \tikzstyle{syncdev}= [item, draw=OliveGreen, fill=green!10, text width=2.5cm]
      \tikzstyle{ram}    = [cylinder, aspect=0.25, inner sep=3mm, shape border rotate=90, draw=Emerald, fill=Emerald!10, text centered, font=\footnotesize]
      \tikzstyle{empty}  = [fill=none, draw=none, inner sep=0mm]

      % Блоки (c = channel, cIJ - канал номер I, блок номер J)
      % Задаём вручную начальные позиции каналов
      \node[empty] (c10) {};
      \node[empty] (c20) [below=3.5cm of c10] {};
      \node[empty] (cN0) [below=3.5cm of c20] {};
      % Всё остальное рисуем в цикле:
      \foreach \x in {1, 2, N} {
        \node[block] (c\x1) [right=0.5cm of c\x0] {Масштабный преобразователь};
        \node[block] (c\x2) [right=0.7cm of c\x1] {Устройство фильтрации};
        \node[block] (c\x3) [right=0.7cm of c\x2] {Аналогово-цифровой \mbox{преобразователь}}; % mbox чтобы избежать переноса
        \node[title] (title\x) at (c\x2.north) {Измерительный канал №$\x$};
        \foreach \from/\to in {c\x0/c\x1, c\x1/c\x2, c\x2/c\x3}
          \draw [arrow] (\from) -- (\to);
        % Узлы под боками (u = under, uIJ)
        \foreach \y in {1,2} {
          \node[empty] (u\x\y)[below=3mm of c\x\y] {};
        }
        % u3 отдельно: он ниже
        \node[empty] (u\x3)[below=6mm of c\x3] {};
        \node[fit=(c\x1)(c\x3)(u\x1)(title\x),around]{\hspace*{\fill}~}; % hspace - хак, чтобы подавить underfull hbox
      }
      \node[controller] (MCU) [right=2.5cm of c23] {Микроконтроллер};
      % Входы в контроллер: mc1, mc2, mcN
      \foreach \x/\angle in {1/160, 2/180, N/200} {
        \node (mc\x) [left=1cm of MCU.\angle] {};
        \draw [arrow] (c\x3) -| (mc\x.center) -- (mc\x.center -| MCU.west);
      }
      % Линия управления
      \node[circle,fill=Blue,inner sep=1mm] (Control) [right=4mm of u11-|mc1] {};
      \node[above right, Blue, font=\footnotesize] at (Control) {Линия управления};
      \draw [-, cline] (MCU) |- (Control);
      % Линия синхронизации
      \node[circle,fill=Red,inner sep=1mm] (Sync) [left=4mm of u23-|mc2] {};
      \node[below right, Red, font=\footnotesize] at (Sync-|Control) {Линия синхронизации};
      \draw [-, sline] (MCU) |- (Sync);
      % Соединяем линии синхронизации и управления с блоками
      \foreach \x in {1,2,N} {
        % Стрелки линии управления
        \foreach \y in {1,2} {
          \draw [->, cline] (Control) |- (u\x\y.center) -- (c\x\y);
        }
        % Стрелки линии синхронизации
        \draw [->, sline] (Sync) |- (u\x3.center) -- (c\x3);
      }
      % Устройство синхронизации
      \node[syncdev] (SyncDev) [below=2cm of MCU] {Устройство синхронизации};
      \draw[<-, thick, OliveGreen] (MCU.340) -- ++(4mm,0mm) |- (SyncDev);
      % ОЗУ
      \node[ram] (RAM) [above=2.2cm of MCU] {ОЗУ};
      \draw[<->, thick, Emerald] (MCU.20) -- ++(4mm, 0mm) |- (RAM);
      % Выход
      \draw[arrow] (MCU.east) -- ++(1, 0);

    \end{tikzpicture}
  \end{myfigure}

  Данные от датчиков физических величин поступают на масштабные преобразователи измерительных
  каналов. Использование при этом микроконтроллера с обратной связью позволяет организовывать
  процесс прецизионного измерения входных величин за счёт разделения информации по каналам
  измерения с разным коэффициентом усиления и её сложением в микроконтроллере, в этом случае часть
  измерительных каналов работает с большим сигналом, а часть настроена на малый сигнал, что
  позволяет измерять аналоговые сигналы в большом диапазоне значений, при этом на выходе
  микроконтроллера мы получаем непрерывный сигнал во всем диапазоне значений физических датчиков.
  Устройства фильтрации также управляются от микроконтроллера, что позволяет не только выполнять
  стандартные операции фильтрации сигнала, но и, применяя специализированные алгоритмы обработки
  информации, оперативно реагировать на изменяющиеся помехи и оперативно адаптировать параметры
  фильтрации, что существенно расширяет возможности всей автоматизированной системы и увеличивает
  общую помехоустойчивость процесса измерения сигналов.

  Отфильтрованные данные поступают на аналого-цифровые преобразователи, которые, в свою очередь,
  управляются по линии синхронизации. Это позволяет решить задачу синхронной дискретизации сигналов
  по времени, при этом точность синхронизации будет определяться, в первую очередь, точностью
  устройства синхронизации, в качестве которого могут использоваться как классические источники
  высокоточных синхроимпульсов, так и генераторы синхроимпульсов, построенные с использованием
  глобальной системы позиционирования (GPS).

  Информация, поступающая на входы микроконтроллера в цифровом виде от аналого-цифровых
  преобразователей может как передаваться далее по линиями связи в основную ЭВМ, так и
  накапливаться во временном оперативном запоминающем устройстве. Это позволяет, с одной стороны,
  применять специализированные алгоритмы обработки информации с целью оперативного управления
  масштабными преобразователями и устройствами фильтрации, а с другой~--- за счёт применения
  высокоскоростных параллельных ОЗУ накапливать данные со скоростью превосходящей возможности
  каналов связи. Такое применение гарантирует, что в заданный промежуток времени весь поток
  поступающей информации будет обработан и в дальнейшем передан в ЭВМ для анализа и хранения.

  Применение микроконтроллеров в подготовке информации для передачи по каналам связи, позволяет
  более полно использовать каждый физический канал связи, что в общем случае позволяет как добиться
  большей скорости передачи информации, так и существенно расширить спектр применяемых видов каналов
  связи: от радиоканала до специализированной вычислительной сети. Применение специализированных
  алгоритмов в подготовке данных позволяет добиться гарантированного прохождения информации по
  каналам связи без потерь, а применение алгоритмов шифрования и распределения доступа позволяет
  большему количеству исследователей одновременно участвовать в физических экспериментах, при этом
  гарантируется надёжность управления всей системой.

  Таким образом, применение микроконтроллеров в качестве управляющего устройства сразу в нескольких
  каналах измерения автоматизированных систем позволяет существенно расширить возможности самой
  системы, кардинально повысить помехоустойчивость системы, решить вопрос синхронизации поступающих
  данных и, как результат, добиться повышения эффективности и достоверности процессов обработки
  измерительных сигналов в автоматизированной системе.

  \subsection{Цифровая обработка сигналов}

  Фильтрация представляет собой одну из самых распространённых операций обработки сигналов. Цель
  фильтрации состоит в подавлении помех, содержащихся в сигнале, или в выделении отдельных
  составляющих сигнала, соответствующих тем или иным свойствам исследуемого процесса. Фильтрация
  сигнала заключается в целенаправленном изменении соотношения между различными компонентами спектра
  сигнала.

  В цифровой обработке сигналов в основном применяются цифровые фильтры. Они обладают высокой
  точностью, компактностью. Они гибкие в настройке и лёгкие в изменении. В отличие от аналоговых
  фильтров передаточная функция не зависит от дрейфа характеристик элементов. Но, несмотря на
  вышеперечисленные плюсы, недостатки у данных фильтров все-таки присутствуют. Это, во-первых, трудность работы с
  высокочастотными сигналами: полоса частот ограничена частотой Найквиста, равной половине частоты
  дискретизации сигнала. Во-вторых, трудность работы в реальном времени~--- вычисления должны быть завершены в
  течение периода дискретизации. Для большей точности и высокой скорости обработки сигналов
  требуется не только мощный процессор, но и дополнительное, дорогостоящее, аппаратное обеспечение в
  виде высокоскоростных ЦАП и АЦП \cite{glinchenko-digital}.

  Как и при анализе аналоговых сигналов, дискретные сигналы можно представить во временной и
  частотной областях. В настоящее время обработку дискретных сигналов чаще всего проводят в
  частотной области, что диктуется значительными сокращениями объёма цифровой аппаратуры и временем
  обработки. Дискретное преобразование Фурье (ДПФ) по существу представляет собой алгоритм вычисления
  гармонических составляющих спектра по заданным дискретным отчётам аналогового сигнала, что
  значительно сокращает время обработки.  Но дискретное преобразование Фурье не всегда бывает
  удобным \cite{solonina-algorithms}.

  Например, для определения одного коэффициента ДПФ сигнальной последовательности из $N$ отчётов
  необходимо выполнить около $N$ операций умножения на комплексное число и столько же сложений, а для
  нахождения всех коэффициентов объем вычислений составит $N^2$. Если длины обрабатываемых массивов
  превышают тысячу единиц, то дискретная спектральная обработка сигналов в реальном масштабе времени
  требует высокопроизводительных вычислительных комплексов.

  Многократно сократить число операций позволяет быстрое преобразование Фурье (БПФ), обеспечивающее
  вычисление коэффициентов ДПФ за меньшее число операций. В основу БПФ положен принцип разбиения
  заданной последовательности отчётов дискретного сигнала на несколько промежуточных
  последовательностей. Для этого число отчётов $N$ разделяется на множители. Затем, определяются
  спектры этих промежуточных последовательностей, и через них находится спектр всего сигнала. В
  зависимости от состава, числа и порядка следования указанных множеств можно создать различные
  алгоритмы БПФ. В цифровой технике удобно обрабатывать сигнальные последовательности со значениями
  $N$, являющимися степенью числа $2$. Это позволяет многократно делить входную
  последовательность отчётов на подпоследовательности.

  Дискретное преобразование Фурье отображает переход последовательности вещественных данных в
  комплексную область, где хорошо разработаны методы анализа, существенно облегчающие изучение и
  трактовку колебательных процессов. Однако обработку вещественных данных желательно выполнять в
  вещественной области. Эту задачу решает дискретное преобразование Хартли (ДПХ), которое, как и
  ДПФ, может применяться в задачах спектрального анализа и цифровой фильтрации.

  Кроме преобразований Фурье и Хартли в практике цифровой фильтрации используются также дискретное
  преобразование Лапласа и Z-пре\-об\-ра\-зо\-ва\-ние, Вейвлеты. Дискретное преобразование Лапласа имеет
  прямую аналогию с преобразованием Лапласа, оно получило широкое распространение благодаря тому, что
  многим соотношениям и операциям над оригиналами соответствуют более простые соотношения над
  изображениями. Так, свёртка двух функций сводится в пространстве изображений к операции умножения,
  а линейные дифференциальные уравнения становятся алгебраическими.

  При исследовании дискретных сигналов и линейных систем, как правило, вместо дискретного
  преобразования Лапласа используют Z-пре\-об\-ра\-зо\-ва\-ние, которое получается из дискретного
  преобразования Лапласа в результате замены переменных.

  Перечисленные алгоритмы обработки сигнала являются наиболее распространёнными, дающие качественный
  результат при относительно простой реализации. Таким образом, выбор алгоритма ЦОС является
  ключевым моментом определяющим достижение поставленной цели \cite{zubarev-realtime}. Реализация
  программного блока ЦОС определяется как выбором языка программирования с учётом необходимой
  скорости обработки, так и заданными аппаратными возможностями вычислительного процессора. Это
  определяется тем, что как при измерении уровня первой гармоники, так и при анализе спектра сигнала
  определяющим является качество цифровой фильтрации гармоник сигнала от первой до сороковой.

  \subsection{Устройство синхронизации}\label{ssec:sync}

  Точная синхронизация по времени частотно-временных и энергетических процедур обработки
  сигналов возможна за счёт применения вычислительных процессоров и специализированных
  инструментальных средств. Решение этой задачи позволяет синхронизировать отдельные процессы
  обработки информации по времени, что является существенным фактором возможности применения
  цифровой обработки сигналов в специализированных областях радиофизики и геофизики \cite{stupin-methods}.

  Анализ возможных путей решения задачи синхронизации, показал, что решить эту задачу возможно как
  классическими источниками высокоточных синхроимпульсов, так и генераторами синхроимпульсов,
  построенные с использованием глобальной системы позиционирования (GPS система). Использование GPS
  системы позволяет добиться точности межканальной синхронизации не менее 100 нс \cite{lombardi-gps}, причём в случае
  распределённого в пространстве объекта измерения нет необходимости направлять синхроимпульсы из
  единого устройства синхронизации, а можно использовать с каждым территориальным измерительным
  каналом своё устройство синхронизации, причём точность синхронизации при этом не изменяется,
  одновременно отсутствует погрешность обусловленная прохождением синхроимпульсов по длинным
  распределённым линиям.

  \subsection{Существующие программные решения}\label{ssec:existing}

  При решении задачи цифровой обработки сигналов программная часть зачастую реализуется отдельно под
  каждую конкретную задачу. % TODO цитата? Более точная формулировка?
  Тем не менее, существуют различные универсальные программные решения, которые можно разбить на две
  основных группы:
  \begin{enumerate}
    \item SCADA-системы (англ. supervisory control and data acquisition — диспетчерское управление и
      сбор данных), программные пакеты, предназначенные для разработки или обеспечения работы в
      реальном времени систем сбора, обработки, отображения и архивирования информации об объекте
      мониторинга или управления. % TODO более точное определение, источник?
      Примеры таких систем: InTouch, TraceMode и др. % TODO ссылки на продукты?
    \item Платформы графического программирования, такие как LabVIEW, MatLab/Simulink и др.
  \end{enumerate}

  Главными преимуществами SCADA-систем являются их высокая производительность и надёжность,
  определяющиеся областью применения этих систем для контроля за крупными промышленными объектами.
  Это обуславливает и их главные недостатки: высокую стоимость и громоздкость, а~также, в отдельных
  случаях, необходимость использования специального оборудования.

  Платформы второй группы лишены этого недостатка. Однако, в отличие от SCADA-систем, они не
  представляют из себя готовые решения, будучи лишь мощными средами для разработки таковых.
  И хотя наличие специализированных инструментов и готовых блоков делает разработку существенно
  проще по сравнению с языками программирования общего назначения, такие системы лучше всего
  подходят для прототипирования и демонстрационных целей, так как имеют худшее быстродействие по
  сравнению с первой группой из-за использования интерпретаторов.

  \todo{больше авторитетных ссылок в весь этот раздел}.

  \section{Постановка задачи}\label{sec:task}

  Целью работы является повышение эффективности и достоверности процессов автоматизированной
  обработки измерительных сигналов. С этой целью разрабатывается специальное программное
  обеспечение, универсальная информационная
  система обработки прецизионных сигналов, % TODO определение прецизионности в данном случае
  являющаяся компромиссом между двумя группами имеющихся решений, описанных в
  подразделе~\ref{ssec:existing}, то есть готовой к интеграции, гибкой системой, но не такой
  громоздкой и сложной, как SCADA-системы.

  К системе предъявляются следующие требования.

  Система должна быть готовой к использованию программой, не требующей модификации программного кода
  для применения к реальной измерительной системе. В частности, система должна реализовывать
  получение данных по протоколу RS-232\cite{sweet-serial} от микроконтроллера, выполняющего оцифровку.

  Необходимо, чтобы быстродействие системы было достаточным для обработки поступающих цифровых
  сигналов в реальном времени. В частности, система должна обрабатывать в секунду не менее 1000
  (\todo{сколько?}) отсчётов, поступающих на вход.

  Для обеспечения точности привязки по времени система должна осуществлять синхронизацию времени с
  использованием системы GPS, что было объяснено выше, в подразделе~\ref{ssec:sync}.

  Система должна иметь два режима работы: диалоговый, в~котором пользователь может управлять
  программой с помощью графического интерфейса и наблюдать изменения обрабатываемых данных,
  визуализируемых в реальном времени, и фоновый режим, в~котором графический интерфейс отключается
  для экономии ресурсов.

  Наконец, для эффективного использования вычислительной мощности многоядерных центральных
  процессоров, получающих в наши дни всё большее распространение~\cite{steam-hardware}, нужно, чтобы
  вычисления могли выполняться параллельно.

  Для достижения цели ставятся следующие задачи:
  \begin{enumerate}
    \item Реализация процесса получения данных.
    \item Разработка высокопроизводительного модуля обработки цифровых данных.
    \item Реализация процесса синхронизации.
    \item Разработка подсистемы визуализации.
  \end{enumerate}

  \section{Реализация системы}

  Система реализуется в~виде набора программ, написанных на языке C++. Выбор языка обоснован
  компромиссом между быстродействием и удобством разработки и интеграции, которые обеспечивает
  использование объектно-ориентированной парадигмы. Система состоит из следующих программ.

  \begin{enumerate}
    \item Основная программа, реализующая получение, обработку, сохранение и визуализацию данных.
    \item Программа статистической обработки накопленных данных, используемая для визуализации
      файлов, сохранённых основной программой.
    \item Вспомогательная программа для генерации тестовых данных, описанная в разделе~\ref{sec:testing}.
  \end{enumerate}

  В последующих подразделах будет описана архитектура этих программ и технологии, используемые при их
  разработке.

  \subsection{Архитектура системы}

  \subsection{Используемые технологии}

  \section{Тестирование системы}\label{sec:testing}

  \section{Результаты}

  % Про ProcessData: (Свидетельство о гос. регистрации программы № 2013612036, Программа ProcessData).


  %\section{Дальнейшая работа}

  \sectiontoc{Заключение}

  В данной работе ставилась цель разработать программную систему обработки прецизионных сигналов,
  работающую в реальном времени и обеспечивающую помимо обработки точную синхронизацию по времени и
  визуализацию обрабатываемых данных. Эта цель была достигнута, и были удовлетворены поставленные в
  разделе~\ref{sec:task} требования, включая высокое быстродействие и гибкость, а~также применение
  параллельных вычислений.

  Проведено тестирование с использованием виртуального последовательного порта и вспомогательной
  программы, передающей по нему заданные данные. В качестве тестовых данных использовались реальные
  данные с сейсмических датчиков. Тестирование показало пригодность системы для реальных измерений.

  Была также разработана и зарегистрирована программа для статистической обработки накопленных
  данных, которая может быть использоваться и как часть системы, и отдельно от неё, например, в
  лабораторных работах по физике.

  Также были исследованы возможности дальнейшего развития и улучшения системы. В частности,
  планируется расширение набора поддерживаемых протоколов для получения данных и добавление новых
  операций обработки данных.

  \begin{flushleft}
    \bibliographystyle{../../biblio/gost705}
    \bibliography{../../biblio/my}
  \end{flushleft}
\end{document}
